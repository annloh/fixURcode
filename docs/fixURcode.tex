\PassOptionsToPackage{unicode=true}{hyperref} % options for packages loaded elsewhere
\PassOptionsToPackage{hyphens}{url}
%
\documentclass[]{book}
\usepackage{lmodern}
\usepackage{amssymb,amsmath}
\usepackage{ifxetex,ifluatex}
\usepackage{fixltx2e} % provides \textsubscript
\ifnum 0\ifxetex 1\fi\ifluatex 1\fi=0 % if pdftex
  \usepackage[T1]{fontenc}
  \usepackage[utf8]{inputenc}
  \usepackage{textcomp} % provides euro and other symbols
\else % if luatex or xelatex
  \usepackage{unicode-math}
  \defaultfontfeatures{Ligatures=TeX,Scale=MatchLowercase}
\fi
% use upquote if available, for straight quotes in verbatim environments
\IfFileExists{upquote.sty}{\usepackage{upquote}}{}
% use microtype if available
\IfFileExists{microtype.sty}{%
\usepackage[]{microtype}
\UseMicrotypeSet[protrusion]{basicmath} % disable protrusion for tt fonts
}{}
\IfFileExists{parskip.sty}{%
\usepackage{parskip}
}{% else
\setlength{\parindent}{0pt}
\setlength{\parskip}{6pt plus 2pt minus 1pt}
}
\usepackage{hyperref}
\hypersetup{
            pdftitle={fixURcode},
            pdfauthor={Anna Lohmann},
            pdfborder={0 0 0},
            breaklinks=true}
\urlstyle{same}  % don't use monospace font for urls
\usepackage{color}
\usepackage{fancyvrb}
\newcommand{\VerbBar}{|}
\newcommand{\VERB}{\Verb[commandchars=\\\{\}]}
\DefineVerbatimEnvironment{Highlighting}{Verbatim}{commandchars=\\\{\}}
% Add ',fontsize=\small' for more characters per line
\usepackage{framed}
\definecolor{shadecolor}{RGB}{248,248,248}
\newenvironment{Shaded}{\begin{snugshade}}{\end{snugshade}}
\newcommand{\AlertTok}[1]{\textcolor[rgb]{0.94,0.16,0.16}{#1}}
\newcommand{\AnnotationTok}[1]{\textcolor[rgb]{0.56,0.35,0.01}{\textbf{\textit{#1}}}}
\newcommand{\AttributeTok}[1]{\textcolor[rgb]{0.77,0.63,0.00}{#1}}
\newcommand{\BaseNTok}[1]{\textcolor[rgb]{0.00,0.00,0.81}{#1}}
\newcommand{\BuiltInTok}[1]{#1}
\newcommand{\CharTok}[1]{\textcolor[rgb]{0.31,0.60,0.02}{#1}}
\newcommand{\CommentTok}[1]{\textcolor[rgb]{0.56,0.35,0.01}{\textit{#1}}}
\newcommand{\CommentVarTok}[1]{\textcolor[rgb]{0.56,0.35,0.01}{\textbf{\textit{#1}}}}
\newcommand{\ConstantTok}[1]{\textcolor[rgb]{0.00,0.00,0.00}{#1}}
\newcommand{\ControlFlowTok}[1]{\textcolor[rgb]{0.13,0.29,0.53}{\textbf{#1}}}
\newcommand{\DataTypeTok}[1]{\textcolor[rgb]{0.13,0.29,0.53}{#1}}
\newcommand{\DecValTok}[1]{\textcolor[rgb]{0.00,0.00,0.81}{#1}}
\newcommand{\DocumentationTok}[1]{\textcolor[rgb]{0.56,0.35,0.01}{\textbf{\textit{#1}}}}
\newcommand{\ErrorTok}[1]{\textcolor[rgb]{0.64,0.00,0.00}{\textbf{#1}}}
\newcommand{\ExtensionTok}[1]{#1}
\newcommand{\FloatTok}[1]{\textcolor[rgb]{0.00,0.00,0.81}{#1}}
\newcommand{\FunctionTok}[1]{\textcolor[rgb]{0.00,0.00,0.00}{#1}}
\newcommand{\ImportTok}[1]{#1}
\newcommand{\InformationTok}[1]{\textcolor[rgb]{0.56,0.35,0.01}{\textbf{\textit{#1}}}}
\newcommand{\KeywordTok}[1]{\textcolor[rgb]{0.13,0.29,0.53}{\textbf{#1}}}
\newcommand{\NormalTok}[1]{#1}
\newcommand{\OperatorTok}[1]{\textcolor[rgb]{0.81,0.36,0.00}{\textbf{#1}}}
\newcommand{\OtherTok}[1]{\textcolor[rgb]{0.56,0.35,0.01}{#1}}
\newcommand{\PreprocessorTok}[1]{\textcolor[rgb]{0.56,0.35,0.01}{\textit{#1}}}
\newcommand{\RegionMarkerTok}[1]{#1}
\newcommand{\SpecialCharTok}[1]{\textcolor[rgb]{0.00,0.00,0.00}{#1}}
\newcommand{\SpecialStringTok}[1]{\textcolor[rgb]{0.31,0.60,0.02}{#1}}
\newcommand{\StringTok}[1]{\textcolor[rgb]{0.31,0.60,0.02}{#1}}
\newcommand{\VariableTok}[1]{\textcolor[rgb]{0.00,0.00,0.00}{#1}}
\newcommand{\VerbatimStringTok}[1]{\textcolor[rgb]{0.31,0.60,0.02}{#1}}
\newcommand{\WarningTok}[1]{\textcolor[rgb]{0.56,0.35,0.01}{\textbf{\textit{#1}}}}
\usepackage{longtable,booktabs}
% Fix footnotes in tables (requires footnote package)
\IfFileExists{footnote.sty}{\usepackage{footnote}\makesavenoteenv{longtable}}{}
\usepackage{graphicx,grffile}
\makeatletter
\def\maxwidth{\ifdim\Gin@nat@width>\linewidth\linewidth\else\Gin@nat@width\fi}
\def\maxheight{\ifdim\Gin@nat@height>\textheight\textheight\else\Gin@nat@height\fi}
\makeatother
% Scale images if necessary, so that they will not overflow the page
% margins by default, and it is still possible to overwrite the defaults
% using explicit options in \includegraphics[width, height, ...]{}
\setkeys{Gin}{width=\maxwidth,height=\maxheight,keepaspectratio}
\setlength{\emergencystretch}{3em}  % prevent overfull lines
\providecommand{\tightlist}{%
  \setlength{\itemsep}{0pt}\setlength{\parskip}{0pt}}
\setcounter{secnumdepth}{5}
% Redefines (sub)paragraphs to behave more like sections
\ifx\paragraph\undefined\else
\let\oldparagraph\paragraph
\renewcommand{\paragraph}[1]{\oldparagraph{#1}\mbox{}}
\fi
\ifx\subparagraph\undefined\else
\let\oldsubparagraph\subparagraph
\renewcommand{\subparagraph}[1]{\oldsubparagraph{#1}\mbox{}}
\fi

% set default figure placement to htbp
\makeatletter
\def\fps@figure{htbp}
\makeatother

\usepackage{booktabs}
\usepackage[]{natbib}
\bibliographystyle{apalike}

\title{fixURcode}
\author{Anna Lohmann}
\date{2020-06-10}

\begin{document}
\maketitle

{
\setcounter{tocdepth}{1}
\tableofcontents
}
\hypertarget{eat-sleep-debug-repeat}{%
\chapter{eat(), sleep(), debug(), repeat()}\label{eat-sleep-debug-repeat}}

\includegraphics[width=0.5\textwidth,height=\textheight]{img/invite.png}

\hypertarget{error-messages}{%
\chapter{Error messages}\label{error-messages}}

The very obvious first step when your code is not working:
Read the error message. If that does not help. \emph{Read it again.}
Often you can also google the error message.
If you do so try eliminating the parts that seem unique to your own code (e.g.~specific dimensions, object names,\ldots{})

\textbf{Pro Tip} (for new or niche packages):

If google won't find the error try \emph{github}.
You might find the error in the package code or in a github issue.

\hypertarget{decifer-the-error-message}{%
\section{Decifer the error message}\label{decifer-the-error-message}}

Asking yourself the following question can help decifer the error message:

\begin{itemize}
\tightlist
\item
  What object is the error message referring to?\\
\item
  What property of the object might be causing the error?\\
  Is the object

  \begin{itemize}
  \tightlist
  \item
    \ldots{} missing?\\
  \item
    \ldots{} of the wrong type?\\
  \item
    \ldots{} the wrong size?\\
  \item
    \ldots{} wrong content?\\
  \item
    \ldots{} misspelled?
  \end{itemize}
\end{itemize}

\hypertarget{code-doesnt-work-without-errors}{%
\section{Code doesn't work without errors}\label{code-doesnt-work-without-errors}}

\hypertarget{raise-warnings-or-messages-to-errors}{%
\subsection{Raise warnings or messages to errors}\label{raise-warnings-or-messages-to-errors}}

You can turn warnings and messages into errors with the foolowing code:
\texttt{options(warn\ =\ 2)}

Promote messages to errors:\\
\texttt{rlang::with\_abort(f(),\ "message")}

\hypertarget{no-visible-warnings-or-messages}{%
\subsection{No visible warnings or messages}\label{no-visible-warnings-or-messages}}

If the malfunctioning code does not produce any errors or warnings check the documentation to see whether some warnings or error messages are turned off by default.
Often this is some form of \texttt{print\ =\ FALSE}, \texttt{verbose\ =\ FALSE} or \texttt{silent\ =\ TRUE}.

\hypertarget{dont-assume-anything}{%
\chapter{Don't assume anything!}\label{dont-assume-anything}}

\begin{itemize}
\tightlist
\item
  Check the variable or file names.
\item
  Check the object content and type.
\item
  Check code that you think was working fine just a minute ago and you are certain you haven't touched.
\item
  Check whether the online example you modeled your code after actually works.
\item
  Check the function documentation, whether it really works the way you thought it would.
\item
  Write a test to confirm expected behaviour.
\end{itemize}

\hypertarget{read-the-documentation}{%
\chapter{Read the documentation}\label{read-the-documentation}}

Reading documentation is a learned skill.
It doesn't come naturally and take a lot of time to master.
This is not always the user's fault.
Also writing documentation is a skill that not all package developers master equally well.

Try always looking for the following elements when reading documentations.

\begin{enumerate}
\def\labelenumi{(\arabic{enumi})}
\tightlist
\item
  What arguments does the function take?
\item
  Which of those arguments have to be provided?
\item
  If they don't have to be provided is the default what I want?
\item
  Which object types and dimensions should the input have?
\item
  Does the input have to be named?
\end{enumerate}

If the documentation is unclear in these regards try inferring some of the
information from examples or a package vignette.

If the information on a specific package is scarce.
Try the github search to see it in action.

\hypertarget{most-common-error-sources}{%
\chapter{Most common error sources}\label{most-common-error-sources}}

In no particular order these are the most important reasons something goes
wrong in my code:

\begin{itemize}
\tightlist
\item
  The object is of the wrong type (i.e.~a matrix instead of a data.frame or the other way round)
\item
  The object should have names but doesn't.
\item
  I forgot to turn of \texttt{stringsaAsFactors\ =\ TRUE}
  (Problem of the past with R 4.0).
\item
  Some object does not exist (typo or silent error in earlier code).
\item
  A \texttt{(}, \texttt{\{}, \texttt{,}, \texttt{"} is missing or misplaced.
\item
  An object from the environment is used instead of a function parameter.
\item
  The object you are looking for is at a different level of a nested list.
\item
  The method can't handle missing data.
\item
  A namespace conflict calls the wrong function.
\end{itemize}

Tipp:
When copy \& paste stuff paste it in a text editor first to
avoid weird non-code symbols.

\hypertarget{check-the-function-code}{%
\chapter{Check the function code}\label{check-the-function-code}}

If you have consulted the documentation and the internet and still have no
idea why your code is not doing what you expect it to do it can be
helpful to check the function definition.

You can get the code of a given function in different ways.
- Click on the function name while pressing shift.
- Typing an empty function call in the console

There are a few things that have to be kept in mind:
(1) The package containing the function has to be loaded.
(2) The function you would like to explore is exported by that package.
(3) It is not a generic method.

\begin{enumerate}
\def\labelenumi{(\arabic{enumi})}
\setcounter{enumi}{1}
\item
  can be solved by adding \texttt{packagename::} in front of the function call.
  That way you can also explore functions that are not exported by a package.
\item
  Can be solved by addding the class in front of the generic function.
  So if you would like to explore the plot method of the bootnet package,
  you would have to call it as \texttt{plot.bootnet()} in order so see the definition.
\end{enumerate}

\hypertarget{s3-and-debugging}{%
\section{S3 and Debugging}\label{s3-and-debugging}}

\begin{itemize}
\tightlist
\item
  \texttt{UseMethod} errors
\item
  The corresponding functions are called \texttt{\textless{}function.class\textgreater{}} or
\item
  \texttt{\textless{}function.default\textgreater{}}
\item
  Use methods() to see what methods are associated with a function or class
\end{itemize}

Press ctrl + left mouse click on the function name to view the function definition

\begin{Shaded}
\begin{Highlighting}[]
\KeywordTok{reverseLetters}\NormalTok{()}
\end{Highlighting}
\end{Shaded}

\hypertarget{object-properties}{%
\chapter{Object properties}\label{object-properties}}

\texttt{str()}

\texttt{dplyr::glimpse()}

\texttt{class()}

\texttt{typeof()}

\texttt{names()}

\texttt{length()}

\texttt{dim()}

\texttt{print()}

\texttt{View()}

\texttt{head()}

\texttt{identical()}

\hypertarget{debugging-strategies}{%
\chapter{Debugging strategies}\label{debugging-strategies}}

\hypertarget{isolate-the-problem}{%
\section{Isolate the problem}\label{isolate-the-problem}}

\hypertarget{minimum-working-example}{%
\subsection{Minimum working example}\label{minimum-working-example}}

\begin{itemize}
\tightlist
\item
  Remove as much code as possible but have the problem still occurring.
\item
  Locate the source of the error.
\item
  Change one thing at a time.
\end{itemize}

\hypertarget{line-by-line-debugging-stepping-through-the-code}{%
\section{Line-by-line debugging (Stepping through the code)}\label{line-by-line-debugging-stepping-through-the-code}}

\begin{itemize}
\tightlist
\item
  Run the code line by line
\end{itemize}

\hypertarget{rubber-duck-debugging}{%
\section{Rubber duck debugging}\label{rubber-duck-debugging}}

Talk yourself through the code. (Or explain it to your rubber duck.)

\hypertarget{print-debugging}{%
\section{Print debugging}\label{print-debugging}}

The idea of print debugging is to print certain objects or object properties
close to where you expect the error might occur.
Printing might show you that an object has differetn properties from what you
would expect it to have or figure our in which iteration or with which seed
something might break.
This is a quick and dirty option to the more elaborate debugging tools.

\hypertarget{logging}{%
\section{Logging}\label{logging}}

\hypertarget{debugging-tools}{%
\chapter{Debugging tools}\label{debugging-tools}}

\hypertarget{interactive-r-session}{%
\subsection{Interactive R session}\label{interactive-r-session}}

\begin{itemize}
\tightlist
\item
  R session reacts to user input
\item
  Check with \texttt{interactive()}
\end{itemize}

\hypertarget{step-through-the-code---debug}{%
\subsection{\texorpdfstring{Step through the code - \texttt{debug()}}{Step through the code - debug()}}\label{step-through-the-code---debug}}

\begin{itemize}
\tightlist
\item
  \texttt{debug(my\_defunct\_function)}
\item
  Call exported functions from a package with \texttt{::}
\item
  Unexported functions can be referred to with \texttt{:::}
\end{itemize}

R will switch to the browser mode every time that function is called from anywhere in R.

You have to tell R to stop debugging \texttt{undebug(my\_defunct\_function)} or overwrite the function (e.g.~by sourcing it).
\texttt{debugonce(my\_defunct\_function)} will go in browse mode the next time the function is called.

\hypertarget{browser}{%
\subsection{browser()}\label{browser}}

\begin{itemize}
\tightlist
\item
  Enter browse() anywhere in your code where you want the execution to halt.
\item
  You can then inspect the environment at that state.
\item
  You can also change objects and continue.
\item
  You can also use \texttt{browser()} as an error handler.
\end{itemize}

\texttt{Browse{[}number{]}} The number tells you at which level of the call stack you are.

\hypertarget{breakpoints}{%
\subsection{Breakpoints}\label{breakpoints}}

\begin{itemize}
\tightlist
\item
  Does the same as ``browse()'' but without lottering your code.
\item
  Click left of a line number for a dot to appear.
\item
  Next time you run the code it will stop there.
\item
  \texttt{options(error\ =\ recover)} automatically switches to browsing when an error occurs
\end{itemize}

\hypertarget{traceback}{%
\subsection{traceback()}\label{traceback}}

\begin{itemize}
\tightlist
\item
  Helps locating the problem.
\item
  Shows you the last commands that have been called before the error occured (i.e.~the stack).
\item
  Reverse order, latest command is on top.
\end{itemize}

\texttt{rlang::last\_trace()} is ordered in the opposite way to \texttt{traceback()}

\hypertarget{debugging-other-peoples-code}{%
\section{Debugging other people's code}\label{debugging-other-peoples-code}}

\hypertarget{ask-a-peer}{%
\chapter{Ask a peer}\label{ask-a-peer}}

Sometimes you are so entangled in your code that you cant see the forest for the trees.
What has kept you stuck for days can often be solved within minutes by a fresh pair of eyes.
And even if the peer can't sove your problem.
The process of walking someone else through it might make you come up with a solution yourself.

Don't assume your error is embarrassing.
If it is, your peer will find and fix it within seconds.
If it isn't you can both learn from it.

\hypertarget{it-works-again---great-do-you-know-why}{%
\chapter{It works again? - Great, do you know why?}\label{it-works-again---great-do-you-know-why}}

\begin{itemize}
\tightlist
\item
  A random stacl overflow post fixed it?
\item
  You made some random changes?
\end{itemize}

Take the time to figure out why these fixes work and try making a note what
the solution was. Chances are you'll break it again or run into the
same error later in your code.

\hypertarget{use-git}{%
\chapter{Use Git!}\label{use-git}}

\begin{itemize}
\tightlist
\item
  Create a bugfix branch (or two or three if you try more complex approaches)
\item
  You can easily get back to your starting point
\item
  You don't create new problems by forgetting to change something back in some remote part of the code
  that turned out to not work
\item
  You can keep track of what you already tried
\item
  If an attempt almost got you there and you discover the one thing that was
  missing three hours later you can easily recreate that ``95\% good'' state
\item
  When you fixed the problem you can analyse the solution by looking at what exactly was changed
\end{itemize}

\hypertarget{error-handling}{%
\chapter{Error handling}\label{error-handling}}

Buiding error handling into your own code can get messy really fast.
Just returning an error message or continuing a loop even when one iteration fails
is easy.
But as soon as you want to adapt behaviour depending on error or save the error messages,
things get messy.

This is not always your fault, but sometimes due to messy error documentation of the package developer.

\hypertarget{unit-tests-with-testthat}{%
\chapter{\texorpdfstring{Unit tests with \texttt{testthat}}{Unit tests with testthat}}\label{unit-tests-with-testthat}}

Idea:
- You are working with a package
- The test live in a separate folder \texttt{tests/testthat/}
- Testing is part of your workflow.

Easy examples for tests:
- \texttt{expect\_error()} the code should throw an error
- \texttt{expect\_warning()} the code should throw a warning
- \texttt{expect\_equal(a,\ b)} a and b should be the same (up to numeric tolerance)
- \texttt{expect\_equivalent(a,\ b)} a and b should contain the same values but may have different attributes (e.g., names and dimnames)

If you are not working with a package check out:

\hypertarget{resources}{%
\chapter{Resources}\label{resources}}

\hypertarget{debugging-with-rstudio---jonathan-mcpherson}{%
\subsection{Debugging with RStudio - Jonathan McPherson}\label{debugging-with-rstudio---jonathan-mcpherson}}

\url{https://support.rstudio.com/hc/en-us/articles/205612627-Debugging-with-RStudio}

\hypertarget{advanced-r---hadley-wickham-chapter-22-debugging}{%
\subsection{Advanced R - Hadley Wickham (Chapter 22 Debugging)}\label{advanced-r---hadley-wickham-chapter-22-debugging}}

\url{https://adv-r.hadley.nz/debugging.html}

\hypertarget{what-they-forgot-to-teach-you-about-r---jennifer-bryan-jim-hester}{%
\subsection{What They Forgot to Teach You About R - Jennifer Bryan, Jim Hester}\label{what-they-forgot-to-teach-you-about-r---jennifer-bryan-jim-hester}}

\url{https://rstats.wtf/debugging-r-code.html}

\hypertarget{toy-examples-to-test-your-debugging-skills}{%
\subsection{Toy examples to test your debugging skills}\label{toy-examples-to-test-your-debugging-skills}}

rstd.io/wtf-debugging

\hypertarget{error-handling-1}{%
\subsection{Error handling}\label{error-handling-1}}

\hypertarget{unit-tests}{%
\subsection{Unit tests}\label{unit-tests}}

Advanced R - Hadley Wickham (Chapter Tests)
\url{http://r-pkgs.had.co.nz/tests.html}

\hypertarget{getting-frustrated}{%
\chapter{Getting frustrated?}\label{getting-frustrated}}

\begin{itemize}
\item
  Are you trying the same thing for the third time?
\item
  You can't remember which of the 20 open tabs you have already tried?
\item
  Hitting your computer or cursing at the screen?
\item
  Randomly changing stuff?
\item
  Copy and paste cryptic code snippets from stack overflow?
\item
  Take a deep breath!
\item
  Remind yourself that computers are completely rational.
  Noone is out there to get you! There is a logical reason!
  It didn't just magically break!
\item
  Take a break!
\item
  Be determined to understand the problem.
\item
  Embrace the challenge and the learning experience!
\end{itemize}

\hypertarget{buggy-code-to-test}{%
\chapter{Buggy code to test}\label{buggy-code-to-test}}

\begin{Shaded}
\begin{Highlighting}[]
\NormalTok{usethis}\OperatorTok{::}\KeywordTok{use_course}\NormalTok{(}\StringTok{"rstd.io/wtf-debugging"}\NormalTok{)}
\end{Highlighting}
\end{Shaded}

\bibliography{book.bib}

\end{document}
